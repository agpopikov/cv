%% start of file `template.tex'.
%% Copyright 2006-2015 Xavier Danaux (xdanaux@gmail.com), 2020-2022 moderncv maintainers (github.com/moderncv).
%
% This work may be distributed and/or modified under the
% conditions of the LaTeX Project Public License version 1.3c,
% available at http://www.latex-project.org/lppl/.


\documentclass[12pt,a4paper,russian]{moderncv}        % possible options include font size ('10pt', '11pt' and '12pt'), paper size ('a4paper', 'letterpaper', 'a5paper', 'legalpaper', 'executivepaper' and 'landscape') and font family ('sans' and 'roman')

% moderncv themes
\moderncvstyle{classic}                             % style options are 
\moderncvcolor{burgundy}                               % color options 
\usepackage[russian]{babel}  % FIXME: using spanish breaks moderncv
\usepackage{translator}

% Переопределение имен месяцев в именительном падеже
\newcommand{\monthnameru}[1]{%
  \ifnum#1=1 январь\fi%
  \ifnum#1=2 февраль\fi%
  \ifnum#1=3 март\fi%
  \ifnum#1=4 апрель\fi%
  \ifnum#1=5 май\fi%
  \ifnum#1=6 июнь\fi%
  \ifnum#1=7 июль\fi%
  \ifnum#1=8 август\fi%
  \ifnum#1=9 сентябрь\fi%
  \ifnum#1=10 октябрь\fi%
  \ifnum#1=11 ноябрь\fi%
  \ifnum#1=12 декабрь\fi}

% https://tex.stackexchange.com/questions/310612/moderncv-date-year-on-top-month
\usepackage{datetime}
\def\dates[#1.#2-#3.#4]{\yearabove{\monthnameru{#1}}{#2}--\yearabove{\monthnameru{#3}}{#4}}
\newcommand{\yearabove}[2]{\parbox[t]{10mm}{\centering{#2\par\vspace{-2mm} \tiny{#1}}}}

\setlength{\hintscolumnwidth}{3cm} % change the width of the column with the dates

% \newfontfamily\bodyfont{Fira Sans}
% \newfontfamily\thinfont[]{Fira Sans Light}
% \newfontfamily\headingfont[]{Fira Sans Bold}

% adjust the page margins
\usepackage[scale=0.75]{geometry}
\setlength{\footskip}{136.00005pt}                 % depending on the amount 

\ifxetexorluatex
  \usepackage{fontspec}
  \setmainfont{Roboto}
  \usepackage{unicode-math}
  \defaultfontfeatures{Ligatures=TeX}
  % \setmainfont{Latin Modern Roman}
  % \setsansfont{Latin Modern Sans}
  % \setmonofont{Latin Modern Mono}
  % \setmathfont{Latin Modern Math} 
\else
  \usepackage[T1]{fontenc}
  \usepackage{lmodern}
\fi

% document language
% \usepackage[english]{babel}  % FIXME: using spanish breaks moderncv


% personal data
\name{Андрей}{Попиков}
\title{Software Engineer}
\address{Воронеж, Россия}
\phone[mobile]{+7~960~103~2074}
\email{agpopikov@gmail.com}
\social[linkedin]{agpopikov}
\homepage{agpopikov.info}
\social[telegram]{agpopikov}

\photo[80pt][0.4pt]{me.min}
\renewcommand*{\bibliographyitemlabel}{[\arabic{enumiv}]}

%----------------------------------------------------------------------------------
%            content
%----------------------------------------------------------------------------------
\begin{document}
%\begin{CJK*}{UTF8}{gbsn}                          % to typeset your resume in 
\makecvtitle
\cvitem{}{Я разработчик с 10+ годами опыта. Моей основной специализацией является разработка ПО на Java и Kotlin (активно использую Kotlin для backend-разработки начиная с версии 1.2). У меня есть большой опыт работы с такими фреймворками и библиотеками как Spring, Hibernate, jOOQ, testcontainers и многими другими. Помимо этого у меня есть опыт формирования и управления командами разработки ПО.}{}
\section{Навыки}
\cvitem{}{Kotlin, Java, Spring Framework (Boot, Security, Cloud), jOOQ, Flyway, PostgreSQL, Apache Cassandra, Clickhouse, Kafka, testcontainers, API design (Swagger / OpenAPI standards), Angular, TypeScript, архитектура ПО, анализ производительности ПО (async-profiler, YourKit), cloud technologies (в основном AWS) включая гибридные решения, формирование и управление командами разработки, гибкие методологии разработки ПО (Kanban, XP, Agile). }

\section{Опыт}

\cventry{\dates[01.2020-08.2024]}{Lead Software Engineer / co-founder}{Madtest}{}{}{%
  \begin{itemize}
    \item Принял участие в создании Madtest на ранней стадии, отвечал за backend-разработку и инфраструктуру.
    \item Разработал архитектуру и большую часть существующего решения, включая авторизацию, платежи, рассылки и др. Используем Kotlin в качестве основного языка для backend-разработки вместе со Spring, PostgreSQL, Kafka, Clickhouse.
  \end{itemize}
}
\cventry{\dates[12.2021-08.2023]}{Lead Software Engineer}{Технолаб (contractor / part-time)}{}{}{%
    \begin{itemize}
    \item Подключился в качестве ведущего инженера для решения проблем с производительностью и улучшения архитектуры одной из крупнейших платформ по продаже билетов в регионе.
    \item Руководил созданием и интеграцией новой команды разработчиков с целью повышения масштабируемости и надежности платформы.
    \item Работал над существующими сервисами, написанными на Java (с использованием Spring, Hibernate), тесно взаимодействовал с DevOps-командой для улучшения поддерживаемости всего решения.
  \end{itemize}
}
\cventry{\dates[08.2020-10.2021]}{Sr. Java Developer}{Айконсофт}{}{}{%
  \begin{itemize}
    \item Работал над разработкой новых и поддержка существующих веб-сервисов на Java (Spring Boot / JHipster).
    \item Занимался внедрением и интеграцией Camunda BPMN как платформы для управления и настройки бизнес-процессов.
  \end{itemize}
}
\cventry{\dates[02.2015-12.2019]}{Lead Software Engineer}{Лига Цифровой Экономики}{Воронеж}{}{%
  \begin{itemize}
    \item Разработка веб-сервисов (в том числе высоконагруженных) с использованием Java, Kotlin, Spring, PostgreSQL, Redis, Kafka и UI-интерфейсов на Angular (TypeScript).
    \item Управлял командой разработки (около 10 разработчиков).
    \item Работал над внедрением и совершенствованием процессов CI/CD, внедрил и стандартизовал практики code review в нескольких командах.
    \item Учавствовал в разработке прототипов с использованием различных технологий (IoT-проекты, мобильные приложения на Flutter и др.) в рамках pre-sale активностей.
    \item Активно участвовал в HR-процессах:
      \begin{itemize}
        \item Принимал участие в технических интервью, работал над материалами и методикой для проведения интервью.
        \item Организовал и провел несколько курсов по основам Java-разработки и Spring для студентов.
      \end{itemize}
  \end{itemize}
}
\cventry{\dates[02.2014-01.2015]}{Java-разработчик}{Вистар}{Воронеж}{}{%
  \begin{itemize}
    \item Разработка веб-сервисов на Java (Spring, Hibernate / MyBatis) для отслеживания общественного транспорта в реальном времени.
    \item Разработка прототипов мобильных приложений с использованием Apache Cordova (Angular.js / jQuery).
  \end{itemize}
}

\section{Языки}
\cvitemwithcomment{Русский}{родной}{}
\cvitemwithcomment{Английский}{свободный (B2/C1)}{}
\cvitemwithcomment{Немецкий}{начинающий (A2)}{}

\section{Образование}
\cvitem{2012--2016}{
  \textbf{Воронежский Государственный Университет}; \newline
  Изучал компьютерные науки @ Факультет компьютерных наук; \newline
  Кафедра Программирования и Информационных Технологий (ПиИТ);
}

\section{Хобби и интересы}
\cvitem{}{Разработка IoT-решений (в основном на основе чипов STM32/ESP32). Есть небольшой опыт использования DSP/FPGA (семейство AMD / Xilinx Zynq) и проектирования простых плат.}
\cvitem{}{}

\clearpage
\end{document}

