\documentclass[11pt,a4paper,sans]{moderncv}        % possible options include font size ('10pt', '11pt' and '12pt'), paper size ('a4paper', 'letterpaper', 'a5paper', 'legalpaper', 'executivepaper' and 'landscape') and font family ('sans' and 'roman')

\moderncvstyle{casual}                             % style options are 'casual' (default), 'classic', 'banking', 'oldstyle' and 'fancy'
\moderncvcolor{blue}                               % color options 'black', 'blue' (default), 'burgundy', 'green', 'grey', 'orange', 'purple' and 'red'
\usepackage[utf8]{inputenc}
\usepackage[russian]{babel}
\usepackage[scale=0.75]{geometry}
% personal data
\name{Попиков}{Андрей}
\title{Разработчик}                               % optional, remove / comment the line if not wanted
\address{г. Воронеж}{Россия}% optional, remove / comment the line if not wanted; the "postcode city" and "country" arguments can be omitted or provided empty
\phone[mobile]{+7~(960)~103~20~74}                   % optional, remove / comment the line if not wanted; the optional "type" of the phone can be "mobile" (default), "fixed" or "fax"
\email{agpopikov@gmail.com}                               % optional, remove / comment the line if not wanted
\homepage{https://agpopikov.info}                         % optional, remove / comment the line if not wanted
\social[github]{agpopikov}                              % optional, remove / comment the line if not wanted
\social[skype]{andrejpopikov}                               % optional, remove / comment the line if not wanted
\social[linkedin]{agpopikov}                               % optional, remove / comment the line if not wanted
% \photo[64pt][0.4pt]{picture}                       % optional, remove / comment the line if not wanted; '64pt' is the height the picture must be resized to, 0.4pt is the thickness of the frame around it (put it to 0pt for no frame) and 'picture' is the name of the picture file
%----------------------------------------------------------------------------------
%            content
%----------------------------------------------------------------------------------
\begin{document}
%-----       resume       ---------------------------------------------------------
\makecvtitle

\section{Образование}
\cventry{2012--2016}{Факультет Компьютерных наук}{ВГУ}{Воронеж}{\textit{Бакалавр}}
{Кафедра "Программирование и информационные технологии"}


\renewcommand{\listitemsymbol}{-~}

\section{Навыки}


\section{Опыт}
\cventry{2015--2019}{Ведущий разработчик}{Лига Цифровой Экономики}{Воронеж}{}
{
\begin{itemize}
\item Разработка веб-сервисов с использованием технологий - Java, Kotlin, Spring, PostgreSQL, Redis, Kafka;
\item Организация CI - 
\item Управление командой :
  \begin{itemize}
  \item Sub-achievement (a);
  \item Sub-achievement (b), with sub-sub-achievements (don't do this!);
    \begin{itemize}
    \item Sub-sub-achievement i;
    \item Sub-sub-achievement ii;
    \item Sub-sub-achievement iii;
    \end{itemize}
  \item Sub-achievement (c);
  \end{itemize}
\item Разработка прототипов / PoC-решений с целью анализа и внедрения технологий.
\item Участие в HR-процессах компании:
  \begin{itemize}
    \item Проведение технических интервью, совершенствование методик и материалов.
    \item Составление программы и проведение курсов по основам Java-разработки для студентов.
  \end{itemize}
\end{itemize}}

\cventry{2014--2015}{Java-разработчик}{Вистар}{Воронеж}{}{
\begin{itemize}
\item Разработка веб-сервисов с использованием - Java, Spring, MyBatis.
\item Разработка прототипов мобильных приложений с использованием платформы Apache Cordova (Angular.js / jQuery).
\end{itemize}}

\cventry{2014--2015}{Стажёр-разработчик}{Лайконикс}{Воронеж}{частичная занятость}{
\begin{itemize}
\item Участие в развитии существующих проектов на базе платформ .Net и Ruby.
\end{itemize}}

\section{Дополнительные проекты}
\cvlistitem{MadTest (http://madtest.ru/) - проект на этапе закрытого тестирования}{
  {Архитектура сервиса, дизайн API и его реализация, частичное участие в разработке UI}
}
\cvlistitem{WIP: Реализация адаптера Apple HomeKit Accessory Protocol over Bluetooth LE}
% \cvlistitem{Item 2}
% \cvlistitem{Item 3. This item is particularly long and therefore normally spans over several lines. Did you notice the indentation when the line wraps?}

\section{Языки}
\cvitemwithcomment{Русский}{Родной}{}
\cvitemwithcomment{Английский}{Intermediate (cвободное чтение и диалог на технические темы)}{}
\cvitem{Немецкий}{A2}

\section{Интересы}
\cvlistitem{Изучение и применение импульсной и цифровой электроники во встраиваемых решениях, а также программирование микроконтроллеров семейств STM32 / ESP32.}
\cvlistitem{}

\end{document}
