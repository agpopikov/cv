\documentclass[11pt,a4paper,sans]{moderncv}        % possible options include font size ('10pt', '11pt' and '12pt'), paper size ('a4paper', 'letterpaper', 'a5paper', 'legalpaper', 'executivepaper' and 'landscape') and font family ('sans' and 'roman')

\moderncvstyle{casual}                             % style options are 'casual' (default), 'classic', 'banking', 'oldstyle' and 'fancy'
\moderncvcolor{blue}                               % color options 'black', 'blue' (default), 'burgundy', 'green', 'grey', 'orange', 'purple' and 'red'
\usepackage[utf8]{inputenc}
\usepackage[russian]{babel}
\usepackage[scale=0.75]{geometry}
% personal data
\name{Попиков}{Андрей}
\title{Разработчик}                               % optional, remove / comment the line if not wanted
\address{г. Воронеж}{Россия}% optional, remove / comment the line if not wanted; the "postcode city" and "country" arguments can be omitted or provided empty
\phone[mobile]{+7~(960)~103~20~74}                   % optional, remove / comment the line if not wanted; the optional "type" of the phone can be "mobile" (default), "fixed" or "fax"
\email{agpopikov@gmail.com}                               % optional, remove / comment the line if not wanted
\homepage{https://agpopikov.info}                         % optional, remove / comment the line if not wanted
\social[github]{agpopikov}                              % optional, remove / comment the line if not wanted
\social[skype]{andrejpopikov}                               % optional, remove / comment the line if not wanted
\social[linkedin]{agpopikov}                               % optional, remove / comment the line if not wanted
% \photo[64pt][0.4pt]{picture}                       % optional, remove / comment the line if not wanted; '64pt' is the height the picture must be resized to, 0.4pt is the thickness of the frame around it (put it to 0pt for no frame) and 'picture' is the name of the picture file
%----------------------------------------------------------------------------------
%            content
%----------------------------------------------------------------------------------
\begin{document}
%-----       resume       ---------------------------------------------------------
\makecvtitle

\section{Образование}
\cventry{2012--2016}{Факультет Компьютерных наук}{ВГУ}{Воронеж}{\textit{Бакалавр}}
{Кафедра "Программирование и информационные технологии"}

\renewcommand{\listitemsymbol}{-~}

\section{Навыки}
\cvitem{}{
\begin{itemize}
\item Знание алгоритмов и паттернов проектирования, принципов сетевого взаимодействия и прикладной криптографии.
\item Опыт управления командой разработки.
\item Опыт разработки архитектуры высоконагруженных проектов и решений.
\item Хорошие знания платформы и языка Java (Core, Collections, Concurrency), а также языка Kotlin.
\item Опыт в разработке больших проектов на базе таких фреймворков, как Spring (Boot, Security, Web), Hibernate.
\item Хорошие знания веб-фреймворка Angular и современных инструментов разработки веб-приложений (NPM / Yarn, TypeScript, Sass).
\item Опыт использования таких инфраструктурных инструментов, как - Gradle, Docker, TestContainers, Ansible, CI (TeamCity, GitLab CI), Nginx.
\end{itemize}}

\section{Опыт}
\cventry{2015--2019}{Ведущий разработчик}{Лига Цифровой Экономики}{Воронеж}{}
{
\begin{itemize}
\item Управление командой разработки (около 10 человек в команде).
\item Разработка веб-приложений с использованием технологий таких технологий как - Java, Kotlin, Spring, PostgreSQL, Redis, Kafka, TypeScript, Angular.
\item Настройка и совершенствование процессов CI / CD.
\item Разработка прототипов и proof-of-concept решений с целью анализа и внедрения технологий (IoT-проекты, мобильные приложения на базе платформы Flutter).
\item Активное участие в HR-процессах компании:
  \begin{itemize}
    \item Участие в мероприятиях поддержки и развития HR-бренда компании.
    \item Проведение технических интервью, совершенствование методик и материалов.
    \item Составление программы и проведение курсов по основам Java-разработки для студентов.
  \end{itemize}
\end{itemize}}

\cventry{2014--2015}{Java-разработчик}{Вистар}{Воронеж}{}{
\begin{itemize}
\item Разработка веб-сервисов с использованием - Java, Spring, MyBatis.
\item Разработка прототипов мобильных приложений с использованием платформы Apache Cordova (Angular.js / jQuery).
\end{itemize}}

\cventry{2013--2014}{Стажёр-разработчик}{Лайконикс}{Воронеж}{частичная занятость}{
\begin{itemize}
\item Участие в развитии существующих проектов на базе платформ .Net и Ruby.
\end{itemize}}

\section{Дополнительные проекты}
\cvlistitem{https://madtest.ru - Архитектура и инфраструктура сервиса, дизайн API и его реализация, частичное участие в разработке UI}
\cvlistitem{https://github.com/agpopikov/hob - Реализация адаптера Apple HomeKit Accessory Protocol over Bluetooth LE для STM32}

\section{Языки}
\cvitemwithcomment{Русский}{Родной}{}
\cvitemwithcomment{Английский}{Intermediate (cвободное чтение и диалог на технические темы)}{}
\cvitem{Немецкий}{A2}

\section{Интересы и хобби-проекты}
\cvlistitem{Разработка IoT-устройств на базе чипов семейств STM32 и ESP32, в том числе с применением FPGA (Xilinx Zynq) и DSP.}
\cvlistitem{Изучение и применение вычислительной математики в симуляциях физических процессов (FEM, CFD).}
\end{document}
